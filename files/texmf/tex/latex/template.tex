\documentclass[oneside,english,chapters]{smireport}
\usepackage[T1]{fontenc}
\usepackage[latin9]{inputenc}
\usepackage{color}
\usepackage{babel}
\usepackage{array}
\usepackage{float}
\usepackage{graphicx}
\usepackage[dvipsnames]{xcolor}
\usepackage{listings}
\definecolor{codegreen}{rgb}{0,0.6,0}
\definecolor{codegray}{rgb}{0.5,0.5,0.5}
\definecolor{codepurple}{rgb}{0.58,0,0.82}
\definecolor{backcolour}{rgb}{0.95,0.95,0.92}
\lstdefinestyle{mystyle}{
    backgroundcolor=\color{backcolour},
    commentstyle=\color{codegreen},
    keywordstyle=\color{magenta},
    numberstyle=\tiny\color{codegray},
    stringstyle=\color{codepurple},
    basicstyle=\ttfamily\footnotesize,
    breakatwhitespace=false,
    breaklines=true,
    captionpos=b,
    keepspaces=true,
    numbers=left,
    numbersep=5pt,
    showspaces=false,
    showstringspaces=false,
    showtabs=false,
    tabsize=2
}
\lstset{style=mystyle}
\usepackage[unicode=true,
  bookmarks=true,bookmarksnumbered=false,bookmarksopen=false,
breaklinks=false,pdfborder={0 0 0},pdfborderstyle={},backref=false,colorlinks=true]
{hyperref}
\newcommand{\titletext}{TEMPLATE TITLE HERE}
\hypersetup{pdftitle={\titletext},
  pdfauthor={Jason Gutel},
  pdfsubject={SMI},
linkcolor=Blue, urlcolor=blue, citecolor=black}

\makeatletter
\setlength\parindent{0pt} % global \noindent

%%%%%%%%%%%%%%%%%%%%%%%%%%%%%% LyX specific LaTeX commands.
%% Because html converters don't know tabularnewline
\providecommand{\tabularnewline}{\\}

%%%%%%%%%%%%%%%%%%%%%%%%%%%%%% Textclass specific LaTeX commands.
\newcommand\ReleasedDrawing{}
\@ifundefined{ReleasedDrawing}
{
  \newcommand\ReleasedDrawing{}
}

%%%%%%%%%%%%%%%%%%%%%%%%%%%%%% User specified LaTeX commands.
\newcommand{\tabref}{\Tabref}
\newcommand{\figref}{\Figref}


\makeatother
\author{Jason Gutel}

\begin{document}

\renewcommand\HeaderTitle{\titletext}{}

\renewcommand\DocNumber{9XXXX-Y}{}

\renewcommand\DocRev{A}{}

\renewcommand\SVNRevision{\$Rev\$}{}

\renewcommand\TitleBlockTitleLineOne{\titletext}{}

\renewcommand\TitleBlockTitleLineTwo{TEMPLATE Title Line 2}{}

\title{\titletext}

\maketitle
\tableofcontents{}

\listoffigures

\listoftables

\chapter{Revision Notes}
\begin{table}[H]
  \begin{tabular}{|>{\centering}p{0.15\columnwidth}|>{\centering}p{0.85\columnwidth}|}
    \hline
    \textbf{\textsc{Rev}} & \textbf{\textsc{Comments}}\tabularnewline
    \hline
    \hline
    A & Initial Release\tabularnewline
    \hline
  \end{tabular}
  \caption{Revision Table}\label{tab:revision-table}
\end{table}



% EVERYTHING BELOW HERE IS FOR REFERENCE
\chapter{Introduction}
Introduction bold \textbf{Text}.
\newline
\href{https://github.com/gutelfuldead/dotfiles}{Here's a link}

\section{Here's a section}
\begin{center}Section centered italicized \textit{Text}.\end{center}
Here's some code not syntax highlighted
\begin{verbatim}
    int main() {
      printf("Hello World\n\r");
      return 0;
    }
\end{verbatim}
Here's some code syntax highlighted
\begin{lstlisting}[language=C]
    // syntax highlighted
    int main() {
      printf("Hello World\n\r");
      return 0;
    }
\end{lstlisting}
Heres importing some code from a file,
\lstinputlisting[language=C]{example.c}
Heres importing some code from a file between lines [4,20],
\lstinputlisting[language=C, firstline=4, lastline=20]{example.c}

\section{References}
\subsection{table example}
Check out table \ref{tab:uselesstable}

\begin{table}[H]
  \begin{tabular}{|>{\centering}p{0.15\columnwidth}|>{\centering}p{0.15\columnwidth}|>{\centering}p{0.7\columnwidth}|}
    \hline
    \textbf{\textsc{Column A}} & \textbf{\textsc{Column B}} & \textbf{\textsc{Column C}}\tabularnewline
    \hline
    \hline
    I & <3 & LaTeX\tabularnewline
    \hline
    I & <3 & Arch\tabularnewline
    \hline
    I & <3 & Vim\tabularnewline
    \hline
  \end{tabular}
  \caption{Noone loves me}\label{tab:uselesstable}
\end{table}

\subsection{figure example}
Check out figure \ref{fig:image} with an image.

\begin{figure}
  %                                    scale to column , scale to height
  \begin{center}\includegraphics[width=0.25\columnwidth,height=0.25\textheight,keepaspectratio]{./images/smi-logo.png}\end{center}
\caption{some image}\label{fig:image}
\end{figure}

\subsection{link to another section}
I think you'll like Section \ref{sec:landn}.

\section{lists and enumerations}
\label{sec:landn}

Check out this stupid Bullets Figure \ref{fig:bullets} and this other stupid enumeration Figure \ref{fig:enum}!
It's cool to place them in figures so you can capture / reference them.

\begin{figure}[H] % place this figure RIGHT HERE
\begin{itemize}
  \item bullet a
  \begin{itemize}
    \item sub-bullet a
    \item sub-bullet b
  \end{itemize}
  \item bullet b
\end{itemize}
\caption{Bullets Fig}\label{fig:bullets}
\end{figure}

\begin{figure}
\begin{enumerate}
  \setcounter{enumi}{-1} % start counting at zero
  \item start at zero obviously
  \item ???
  \item PROFIT
\end{enumerate}
\caption{stupid enums example}\label{fig:enum}
\end{figure}

\begin{verbatim}
 // Verbatim is used for dropping in code
 int main()
 {
     printf("Hello World\n\r");
     return 0;
 }
\end{verbatim}

% Include some other tex file so it gets compiled into this one
% Secondary file shouldn't have a preamble or a \start{document} \end{document} tags
% Should just have chapters, sections, figures, tables, etc
% Don't include the .tex
\include{secondary-tex-document-name}

\end{document}

